\documentclass[arhiv]{../izpit}
\usepackage{fouriernc}
\usepackage{xcolor}
\usepackage{minted}


\begin{document}

\izpit{Programiranje II: poskusni izpit}{17.\ april 2024}{
  Čas reševanja je 60 minut.
  Veliko uspeha!
}

%%%%%%%%%%%%%%%%%%%%%%%%%%%%%%%%%%%%%%%%%%%%%%%%%%%%%%%%%%%%%%%%%%%%%%%

\naloga[\tocke{10}]

Za vsakega izmed spodnjih programov prikažite vse spremembe sklada in kopice, če poženemo funkcijo \mintinline{rust}{main}. Za vsako spremembo označite, po kateri vrstici v kodi se zgodi.

\podnaloga
\begin{minted}{rust}
  fn f(a: i32, b: i32) -> i32 {
    a * b
  }
  fn g(x: i32) -> i32 {
    f(x, x + 1)
  }
  fn main() {
    let m = 6;
    let n = g(m);
    println!("{n}")
  }
\end{minted}

\podnaloga
\begin{minted}{rust}
  fn f(s: String) {
    println!("{s}")
  }
  fn g(s: String) {
    f(s)
  }
  fn main() {
    let s2 = String::from("2");
    let s1 = String::from("4");
    if true {
      println!("{s2}");
    }
    g(s1);
  }
\end{minted}

\podnaloga
\begin{minted}{rust}
  fn f(s: &String) {
    println!("{s}")
  }
  fn g(s: String) {
    f(&s)
  }
  fn main() {
    let s1 = String::from("4");
    let s2 = String::from("2");
    g(s1);
    println!("{s2}");
  }
\end{minted}


%%%%%%%%%%%%%%%%%%%%%%%%%%%%%%%%%%%%%%%%%%%%%%%%%%%%%%%%%%%%%%%%%%%%%%%

\naloga[\tocke{10}]

Definirajmo tip množic \mintinline{rust}{Set<T>}. Dopolnite signature spodnjih metod. Če v dani prostor ni treba dopisati ničesar, ga prečrtajte.

\podnaloga
\mintinline{rust}{fn contains(_____ self, x: ______) ________}, ki preveri, ali dana množica vsebuje element \mintinline{rust}{x}.

\podnaloga
\mintinline{rust}{fn power_set(_____ self) ________}, ki vrne potenčno množico dane množice.

\podnaloga
\mintinline{rust}{fn intersection(_____ self, other: ______) ________}, ki izračuna presek dveh množic.

\podnaloga
\mintinline{rust}{fn add(_____ self, x: ______) ________}, ki v obstoječo množico doda element \mintinline{rust}{x}.

\podnaloga
\mintinline{rust}{fn into_iter(_____ self) ________}, ki iz množice naredi iterator po njenih elementih.


%%%%%%%%%%%%%%%%%%%%%%%%%%%%%%%%%%%%%%%%%%%%%%%%%%%%%%%%%%%%%%%%%%%%%%%

\naloga[\tocke{30}]

Za vsakega izmed spodnjih programov:
\begin{enumerate}
  \item razložite, zakaj in s kakšnim namenom Rust program zavrne;
  \item program popravite tako, da bo veljaven in bo učinkovito dosegel prvotni namen.
\end{enumerate}

\podnaloga
\begin{minted}{rust}
fn main() {
    let v = vec![1, 2, 3];
    for x in v {
      v.push(x);
    }
}
\end{minted}

\podnaloga
\begin{minted}{rust}
enum Drevo {
    Prazno,
    Sestavljeno(Drevo, u32, Drevo),
}
\end{minted}

\podnaloga
\begin{minted}{rust}
fn daljsi_niz(s1: &str, s2: &str) -> &str {
    if s1.len() > s2.len() {
        return s1;
    } else {
        return s2;
    }
}
\end{minted}

\podnaloga

\begin{minted}{rust}
fn g(s1: &String, s2: &String) -> () {
    /// Poljubna koda, da je tip funkcije ustrezen
}

fn main() {
    let mut s1 = String::from("1");
    g(&mut s1, &s1);
}
\end{minted}

\podnaloga

\begin{minted}{rust}
fn vsebuje<T>(v: &Vec<T>, x : &T) -> bool {
    for y in v {
      if x == y {
        return true
      }
    }
    return false
}
\end{minted}

\end{document}
